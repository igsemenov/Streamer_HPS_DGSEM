\documentclass[fleqn]{article}

\usepackage{amssymb}
\usepackage{amsmath}
\usepackage{latexsym}
\usepackage{url}
\usepackage{xcolor}
\usepackage[implicit=false]{hyperref}
\usepackage{threeparttable}
\usepackage{graphicx}
\usepackage{geometry}
\usepackage{authblk}
\usepackage[square,numbers]{natbib}

\usepackage{sectsty}
\sectionfont{\fontsize{12}{15}\selectfont}

\geometry{twoside,
 paperwidth=192mm,
 paperheight=262mm,
 textheight=635.35pt,
 textwidth=467.75pt,
 inner=37pt,
 outer=42pt,
 top=40pt,
 bottom=55pt,
 headheight=12pt,
 headsep=12pt,
 footskip=16pt,
 footnotesep=28pt plus 2pt minus 6pt,
 columnsep=18pt
}

%--------------------------------------------------------------------------------------------
\begin{document}
%--------------------------------------------------------------------------------------------
\section*{Validation of the FV scheme in 2D}
%--------------------------------------------------------------------------------------------
The computer code implementing the FV scheme is tested by performing a number of experiments. The test computations are performed for the equation of the form
\begin{equation}
\label{EqAd}
\frac{\partial u}{\partial t}+\frac{\partial f^{(x)}}{\partial x}+\frac{\partial f^{(y)}}{\partial y}=q, \quad (x,y) \in [x_{1},x_{2}] \times [y_{1},y_{2}],
\end{equation}
where
\begin{equation}
f^{(x)}=a^{(x)}u-b^{(x)}\frac{\partial u}{\partial x}, \quad
f^{(y)}=a^{(y)}u-b^{(y)}\frac{\partial u}{\partial y}.
\end{equation}
In all cases, the periodic boundary conditions are used. The computational mesh consists of a few block elements. Each block element includes $m \times m$ finite elements ($m$ elements per direction). Each finite element is discretized with a $n \times n$ tensor product grid of FV equidistant nodes.

The goal of these experiments is to test the key components of the FV scheme for different types of interelement contacts. Equation~(\ref{EqAd}) is discretized using the FV scheme combined with the second-order total-variation-diminishing Runge-Kutta scheme [S.\,Gottlieb and C.-W.\,Shu, Mathematics of computation 67 (1998), 73-85].

Two problems are chosen for the experiments:
%--------------------------------------------------------------------------------------------
\section{Problem 1}
%--------------------------------------------------------------------------------------------
This formulation is based on the assumption that $a^{(x)} = a_{x}$, $a^{(y)}=a_{y}$, $b^{(x)}=b^{(y)}=b$, where $a_{x}=const$, $a_{y}=const$, $b=const$.  The initial condition is given by $u(x,y)=\exp (-x^{2}-y^{2} )$ at $t=0$.  The exact solution of this problem is given by
\begin{displaymath}
u(x,y)=\alpha \exp\left[-\alpha \left(x-a_{x}t\right)^{2}-\alpha \left(y-a_{y}t\right)^{2} \right], \quad t>0,
\end{displaymath}
where $\alpha=(4 b \tau)^{-1}$, with $\tau=(4b)^{-1}+t$.  The following mesh configurations are considered: \\

{\bf Test~1.1}. \\ \indent
The mesh includes a single block element $\Omega=[-4,4] \times [-4,4]$. 

{\bf Test~1.2}.\\ \indent
The mesh includes the block elements $\Omega_{1}=[-4,0] \times [-4,0]$, $\Omega_{2}=[0,4]\times [-4,0]$, $\Omega_{3}=[-4,4] \times [0,8]$.

{\bf Test~1.3}.\\ \indent
The mesh includes the block elements $\Omega_{1}=[-4,0] \times [-4,0]$, $\Omega_{2}=[-4,0]\times [0,4]$, $\Omega_{3}=[0,8] \times [-4,4]$. 

%--------------------------------------------------------------------------------------------
\section{Problem 2}
%--------------------------------------------------------------------------------------------

This formulation is designed using the method of manufactured solutions. The transport coefficients are defined as
\begin{displaymath}
a^{(x)}=1+\sin(x+y), \quad a^{(y)}=1+\cos(x+y), \quad
b^{(x)}=0.1+0.1\cos(x+y), \quad b^{(y)}=0.1+0.1\sin(x+y).
\end{displaymath}
The exact solution is assumed to be $u(x,y)=\sin(x+y-t)$. The source term $q$ is derived analytically according to this assumption. The following mesh configurations are considered:\\

{\bf Test~2.1}.\\ \indent
The mesh includes a single block element $\Omega=[-\pi,\pi] \times [-\pi,\pi]$. 

{\bf Test~2.2}.\\ \indent
The mesh includes the block elements $\Omega_{1}=[-2\pi,0] \times [-2\pi,0]$, $\Omega_{2}=[0,2\pi] \times [-2\pi,0]$, $\Omega_{3}=[-2\pi,0] \times [0,2\pi]$, $\Omega_{4}=[0,2\pi] \times [0,2\pi]$ and $\Omega_{5}=[-2 \pi ,2 \pi] \times [-6\pi,-2\pi]$.

{\bf Test~2.3}.\\ \indent
The mesh includes the block elements $\Omega_{1}=[-2\pi,0] \times [-2\pi,0]$, $\Omega_{2}=[0,2\pi] \times [-2\pi,0]$, $\Omega_{3}=[-2\pi,0] \times [0,2\pi]$, $\Omega_{4}=[0,2\pi] \times [0,2\pi]$ and $\Omega_{5}=[2\pi,6\pi] \times [-2 \pi ,2 \pi]$.

%--------------------------------------------------------------------------------------------
\end{document}
%--------------------------------------------------------------------------------------------
